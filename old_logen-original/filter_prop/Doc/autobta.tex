\documentclass[11pt]{article}
\usepackage{graphicx}
\usepackage{amssymb}

\textwidth = 6.5 in
\textheight = 9 in
\oddsidemargin = 0.0 in
\evensidemargin = 0.0 in
\topmargin = 0.0 in
\headheight = 0.0 in
\headsep = 0.0 in
\parskip = 0.2in
\parindent = 0.0in

\newtheorem{theorem}{Theorem}
\newtheorem{corollary}[theorem]{Corollary}
\newtheorem{definition}{Definition}

\title{Automatic Binding Time Analysis Algorithm}
\author{John P. Gallagher \and Kim S. Henriksen}
\begin{document}
\maketitle
\section{Introduction to the BTA problem}
The problem is as follows.  Given a logic program and a goal or set of goals
described by a pattern, type etc, decide on an unfolding rule for offline
partial evaluation.

An unfolding rule, in this context, is defined by an annotation of the program.
Each body atom of the program is marked as follows:
\begin{itemize}
\item
{\em memo}:  do not select the atom for unfolding.
\item
{\em unfold}:  the atom is selected for unfolding whenever it is the leftmost
selectable atom in the resolvent.
\item
{\em call}:  the atom is selected for unfolding whenever it is the leftmost
selectable atom in the resolvent,
and the complete computation tree is constructed (that is, the atom is completely
evaluated).
{\em rescall}:  do not select the atom for unfolding.   This is no
different from {\em memo} as regards unfolding, but there is a difference in
code generation, since atoms marked {\em rescall} are assumed to be defined
outside the program.
\end{itemize}
The main decision to make is therefore between selecting or not selecting
each atom ({\em memo} or {\em rescall} versus {\em unfold} or {\em call}.

For a given atom, the choice of annotation depends on two main considerations.
\begin{itemize}
\item
The state of instantiation of the atom when it is potentially selectable.
\item
The need to ensure termination of unfolding.
\end{itemize}
The first consideration can be solved by fairly standard static analysis of the program with 
respect to the given goal(s).  The second requires an analysis of the termination
behaviour of the program itself (to ensure termination of unfolding) and also the
termination behaviour of the partial evaluation algorithm (to ensure global
termination).

\section{Propagating Call and Success Patterns}

We use the analysis framework described in \cite{Gallagher-Boulanger-Saglam-ILPS95}
and \cite{Codish-Demoen}.  
This consists of the following steps:
\begin{itemize}
\item
Abstract a program to a {\em domain program}, in which every non-variable
term $f(x_1,\ldots,x_n)$ in a clause  (starting with innermost terms)
is replaced by a fresh variable $u$, and an atom
$denotes(f(x_1,\ldots,x_n),u)$ is added to the clause body. 
\item
Supply a definition of the $denotes$ predicate.  This is formally a
pre-interpretation of the function symbols of the program over an abstract
domain $D$.  It can also be seen as a complete bottom-up deterministic
tree automaton in which the set of states is $D$, and the $denotes$ predicate
gives the transitions.
\item
The bottom-up step: Compute the least model of the transformed program together with
the $denotes$ program.  This yields a set of {\em success patterns}
over the abstract domain $D$.
\item
The top-down step: Compute call patterns by using a magic-set transformation,
with
respect to the given goal abstracted in the domain $D$.  The answer
predicates in the magic-set program are the success patterns obtained
from the previous step.  Note that the abstract domain defined by
a $denotes$ predicate is {\em condensing}, which means that the
call patterns computed in this way are as precise as those derived from
a goal-directed analysis.
\item
Use the call patterns to derive a call pattern for each body atom, as
well as a single call pattern for each predicate.
\end{itemize}

\section{Defining the Abstract Domain From Regular Types}

We have developed a novel technique for defining an abstract domain
based on regular types. 
Given some regular types, a {\em determinisation} procedure is applied.
Following this a {\em completion} is derived.  The result is a definition
of types which are (a) disjoint and (b) recognise every term in the
given signature (the set of functions occurring in the program
and goals).

To be described in detail by Kim.....

Typically we start from some standard regular types.  These are the
set of ground terms, and the set of all terms.  These sets are not normally
thought of as types.  A novel aspect of our approach is that it allows
mode information to be integrated seamlessly with type information.

The type consisting of ground terms is called $static$.
The set of all terms, called $dynamic$  including both ground and non-ground terms,
is realised by adding a special constant $v$ with type $var$.
$v$ also has the type $dynamic$.  The constant $v$ does not appear in
any program or goal.  Hence, the only way it can appear in the
argument of a call or success pattern is that if that argument is
possibly variable.  The use of such constants, called {\em no-term
elements} in  \cite{Gallagher-Boulanger-Saglam-ILPS95}
is a way of tracking freeness information in a model-based, declarative
analysis.

The standard types are as follows.

\begin{verbatim}
a -> static.
[] -> static.
[static|static] -> static.
% f(static,static,....,static) -> static  for each f

[] -> list.
[dynamic|list] -> list.

v -> var.

v -> dynamic.
a -> dynamic.
[] -> dynamic.
[dynamic|dynamic] -> dynamic.
% f(dynamic,dynamic,....,dynamic) -> dynamic  for each f
\end{verbatim}
User defined types can then be added to these before the determinsation
process.  Note that the presence of the types $static$ and $dynamic$
ensures that the determinised automaton is also complete.

\section{Derivation of Filters}

The filters are simply the call patterns for each predicate.

\section{Analysis of Annotated Programs}

The process of deriving annotations is an iterative one.
Starting from a program in which every call is marked as unfold,
we derive call patterns and filters.  Some of the unfold annotations
are then changed to {\em memo} (or {\em rescall} if the
derived call pattern is considered too general to be
unfolded.  Then the analysis is re-run.

Each atom marked at memo is simply ignored when computing the
abstract model of the program (the bottom-up step).  Memo-ed
atoms are simply omitted, since they contribute no answers.

In the top-down step, the ``query clauses" in the magic set transformation
are modified so that the answer atoms for memo-ed atoms are omitted.

 \end{document}